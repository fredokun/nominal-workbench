\documentclass[xcolor=dvipsnames]{beamer}

\usepackage[utf8]{inputenc}
\usepackage[french]{babel}
\usepackage{url}
\usepackage{lmodern}
\usepackage{listings}
\usepackage{graphicx}
\usepackage{xcolor}
\usepackage{textcomp} 
\usepackage{hyperref}
%% \usepackage{subcaption}
\usepackage[T1]{fontenc}
\usepackage{graphicx}
\usepackage{verbatim}
\usepackage{color}
 
\setbeamertemplate{footline}{\hspace*{.5cm}\scriptsize{\insertauthor\hspace*{50pt} \hfill\insertframenumber\hspace*{.5cm}}}


\setbeamertemplate{section in toc}{%
\textcolor{MidnightBlue}{$\blacktriangleright$ \inserttocsection}
}

\AtBeginSection[]
{
  \begin{frame}<beamer>{}
    \tableofcontents[currentsection,hideothersubsections,sections={1-5}]
  \end{frame}
} 

\usecolortheme{seahorse}
\usecolortheme{rose}
\useoutertheme{infolines}

\usecolortheme[named=SkyBlue]{structure}
\setbeamercolor{block title}{fg=MidnightBlue}

\title[Nominal Workbench]{Nominal Workbench\\Système de Réécriture Nominale}

\author[\'Equipe NoWork]{Rémy B. Yohan B. Vincent B. Mathieu C. Pierrick C. Matthieu D. Roven G. I\~nigo M. Pierre T.}

\institute[UPMC]{Université Pierre et Marie Curie}

\date{Jeudi 20 mars 2014}

\begin{document}

\begin{frame}
\titlepage
\end{frame}

%% Recette

%% présentation générale du système de réécriture / architecture
%% Rémy
\section{Introduction}
%% présentation générale
%% architecture





\begin{frame}{Introduction}

  \begin{block}{Système de Réécriture}

    Une méthode pour modéliser de façon générale les opérations de
    remplacement de sous-termes par d'autres.
    ~\\

    \emph{ie. etape de réduction pour la semantique d'un language,
      Lois de De Morgan en logique propositionnelle, arithmétique de Peano}
    
  \end{block}

  
\end{frame}

\begin{frame}{Système de Réécriture}

  \begin{block}{Definition}
    
    \begin{itemize}
      \item un ensemble de termes 
      \item un ensemble de règles (terme * terme)
    \end{itemize}

  \end{block}

  réécrire un terme : appliquer les règles conformément à une
  \textcolor{red}{stratégie} prédéfinie sur un term donné en
  \emph{matchant} et en replaçant ses sous-termes.

  ~\\
  Par exemple : \\
  Rule : \texttt{Sub(Add(X))} $\rightarrow$ \texttt{X}\\
  ~\\
  \textcolor{blue}{\texttt{Sub(Add(0))}} $ \rightarrow $ \textcolor{blue}{\texttt{0}}

\end{frame}


\begin{frame}{Réécriture Nominale Non Linéaire}
  
  \begin{block}{Nominale}
    Il existe une distinction entre atomes (variables du langage) et
    méta-variables/placeholders (sous-termes à remplacer).
  \end{block}

  \begin{block}{Non-Linéaire}
    Deux instances d'une même méta-variable peuvent apparaître dans la
    partie gauche de la règle.
  \end{block}
  
\end{frame}



\begin{frame}{\textcolor{red}{No}minal \textcolor{red}{Work}bench}

  L'unique atelier de réécriture qui implante :
  \textcolor{blue}{reconnaissance de motif non-linéaire} +
  \textcolor{blue}{réécriture nominale}

  \begin{block}{Fonctionnalités}
  \begin{itemize}
    \item Définissez des systèmes de réécriture dans \textcolor{red}{un langage simple}
    \item Ecrivez des \textcolor{red}{strategies}
      complexes de façon expressive
    \item Executez des opérations de réécriture sur des termes dans un
      \textcolor{red}{interprèteur intéractif}. 
  \end{itemize}
  \end{block}
  
\end{frame}


\begin{frame}{Architecture des modules (simplifiée)}

\begin{figure}[h]
\begin{center}
\includegraphics[ height=0.9\textheight]{imports/architecture.pdf}
\end{center}
\end{figure}

\end{frame}


\begin{frame}{Arbres de syntaxe abstraite}

\begin{figure}[h]
\begin{center}
\includegraphics[ height=0.7\textheight]{imports/asts.pdf}
\end{center}
\end{figure}

\end{frame}


%% syntaxe/langage du bouzin
%% Matthieu
\section{Langage}
\begin{frame}{Description d'un système de réécriture}
  \begin{block}{Description}
    La description d'un système se fait par des déclarations de 4
    sortes différentes :
    \begin{itemize}
    \item \verb?kind?
    \item \verb?constant?
    \item \verb?operator?
    \item \verb?rule?
    \end{itemize}
  \end{block}
\end{frame}

\begin{frame}[fragile]{Kind}

  Le mot-clé \verb?kind? pour décrire les types manipulés par le
  système de réécriture.
\begin{verbatim}
  kind Integer : type
  kind List : type -> type
  kind Couple : type -> type -> type
  kind Var : atom
\end{verbatim}
\end{frame}

\begin{frame}[fragile]{Constant}
  \begin{itemize}
  \item élément de base du langage
  \item typé avec les types du langage décrit (\verb?kind?)
  \end{itemize}

\begin{verbatim}
  kind Integer : type
  kind Bool : type
  kind List : type -> type
  kind Couple : type -> type -> type

  #generics
  constant Nil : forall(A).List<A>
  constant NilAssoc : forall(A,B).List<Couple<A,B>>

  #instatiated
  constant IntNil : List<Integer>
\end{verbatim}
\end{frame}

\begin{frame}[fragile]{Operator}
  Déclarations similaires aux déclarations de constantes.
\begin{verbatim}
  operator Cons : forall(A).A * List<A> -> List<A>
  operator Hd : forall(A).List<A> -> A
  operator Tl : forall(A).List<A> -> List<A>

  # Hd(Tl(Cons(False, Cons(True, Nil))))
\end{verbatim}
\end{frame}

\begin{frame}[fragile]{Rule}
  \begin{itemize}
  \item L'ordre d'application des règles est définit par les stratégies
  \item Généricité structurelle (\emph{placeholders})
  \end{itemize}

  \begin{verbatim}
    rule [hd]:
      Hd(Cons(?x, ?y)) => ?x

    rule [tl]:
      Tl(Cons(?x, ?y)) => ?y
  \end{verbatim}
\end{frame}


%% type-checking
%% Mathieu Matthieu
\section{Typage}
\begin{frame}{Typage - Bases}

\begin{itemize}
\item Pas de symbole inconnu
\smallskip
\item Arités des applications correctes (kinds, opérateurs)
\smallskip
\item On ne peut pas appliquer un type générique
\smallskip
\item Pas d'atom dans un kind \emph{flèche} %à reformuler
\smallskip
\item Binders de kind atom
\end{itemize}

\end{frame}

\begin{frame}{Typage des règles}

Tous les \emph{placeholders} de l'effet apparaissent dans le pattern
\medskip

On veut parcourir le pattern puis l'effet avec un environnement comprenant :
\smallskip

\begin{itemize}
\item les types des \emph{placeholders}
\item les instanciations de types génériques
\end{itemize}
\bigskip

\textit{Exemple :}
\linebreak
\texttt{operator O : forall(A). A * A -> A}
\linebreak
\texttt{rule [r] : O(C1, C2) => C1}

\end{frame}

\begin{frame}{Typage - Unification}

\textbf{Problème :} gérer les types génériques lors du parcours en profondeur d'un terme
\medskip

\textit{Exemple :}\linebreak
\texttt{operator O : forall(A). A * B -> A}\linebreak
\texttt{rule [r] : O(O(C1,C2), C2) => C1}% avec type(C1) != type (C2)
\medskip

\textbf{Idée :} 
\begin{itemize}
\item l'environnement pour les types génériques n'est valable que pour un niveau
\item notion d'équivalence entre types (\textit{forall(A).A $\approx$ forall(B).B})
\item renommage de type
\end{itemize}

\end{frame}


%% pattern-matching / hash-consing
%% Pierrick
\section{Pattern Matching}

\begin{frame}

\begin{block}{Pattern matching}
Sélectionne une règle de réécriture à partir de la forme d'un terme.
\end{block}

\bigskip

\begin{block}{Types de pattern matching}
\begin{itemize}
\item Linéaire : \texttt{App(?x, ?y)}
\item Non linéaire : \texttt{App(?x, ?x)}
\end{itemize}
\end{block}

\end{frame}

\subsection{Pattern matching linéaire}

\begin{frame}
\frametitle{Exemple de pattern matching}

Soit le terme \texttt{Lambda(x, Var(x))}
\begin{itemize} 

  \item match \texttt{Lambda(\_, \_)}.
  \item mais pas \texttt{Var(\_)}.

\end{itemize}

\bigskip

\begin{block}{Extraction de sous-termes}
\texttt{Lambda(?x, \_)} retourne le sous-terme correspondant à \emph{?x}.
\end{block}

\end{frame}

\begin{frame}
\frametitle{Simulation}

Exemple du déroulement du pattern-matching:
\begin{itemize}
  
  \item Terme : \texttt{Lambda(y, Var(y))}.

  \item Pattern : \texttt{Lambda(\_, Var(?x))}.

\end{itemize}

\bigskip

\begin{center}
  \only<2>
      {\includegraphics[scale=0.5]{pattern/trivial1.pdf}}
  \only<3>
      {\includegraphics[scale=0.5]{pattern/trivial2.pdf}}
  \only<4>
      {\includegraphics[scale=0.5]{pattern/trivial3.pdf}}
  \only<5>
      {\includegraphics[scale=0.5]{pattern/trivial4.pdf}}
  \only<6>
      {\includegraphics[scale=0.5]{pattern/trivial5.pdf}}

\end{center}

\end{frame}

\begin{frame}
\frametitle{Non linéarité des atomes}

Possibilité de matcher deux atomes liés même dans le cas linéaire.

Par exemple avec \texttt{Lambda(?x, Var(?x))} :

\bigskip
\begin{center}
  \only<2>
      {\includegraphics[scale=0.5]{pattern/atom1.pdf}}
  \only<3>
      {\includegraphics[scale=0.5]{pattern/atom2.pdf}}
  \only<4>
      {\includegraphics[scale=0.5]{pattern/atom3.pdf}}
  \only<5>
      {\includegraphics[scale=0.5]{pattern/atom4.pdf}}
  \only<6>
      {\includegraphics[scale=0.5]{pattern/atom5.pdf}}

\end{center}


\end{frame}

\subsection{Pattern matching non linéaire}

\begin{frame}
\frametitle{Non linéarité avec les termes}

\begin{block}{Non linéarité}
Possibilité d'extraire plusieurs fois le même motif modulo alpha-conversion.
\begin{itemize}
\item Atomes : possible dans l'algorithme linéaire.
\item Termes : nécessite la capacité de raisonner sur la structure du terme.
\end{itemize}
\end{block}

\pause

\begin{block}{Exemple}
\begin{itemize}
\item Terme : \texttt{App(Lambda(x, Var(x)), Lambda(y, Var(y)))} match
\item Pattern : \texttt{App(?X, ?X)}.
\end{itemize}
\end{block}

\medskip

Possible puisque le Lambda de droite n'est qu'un renommage du Lambda de gauche.

\end{frame}

\begin{frame}
\frametitle{Solution : hashconsing}

\begin{block}{Principe du hashconsing}
\begin{itemize}
\item Ne pas allouer deux fois deux structures identiques.
\item Deux termes structurellement identiques partagent l'allocation en mémoire.
\item Abstraction des noms de binders et de variables
liées.
\end{itemize}
\end{block}

\end{frame}

\begin{frame}[fragile]
\frametitle{Hashconsing : exemple d'égalité structurelle}

\begin{columns}
  \column{.5\textwidth}
  Pour le terme 
  \begin{verbatim}
  App(
     Lambda(x, Var(x)), 
     Lambda(y, Var(y))
  )
  \end{verbatim}

  \medskip

  \only<1>{
  
    \column{.5\textwidth}
    Avec hashconsing, l'équivalence structurelle apparait :
    \begin{center}
      \includegraphics[scale=0.5]{pattern/pres_hash.pdf}
    \end{center}
  }
  
  \only<2>{
  
    \column{.5\textwidth}
    Sans hashconsing :
    \begin{center}
      \includegraphics[scale=0.4]{pattern/no_hash.pdf}
    \end{center}
  }
\end{columns}
\end{frame}


%% réécriture / stratégie
%% Roven
\section{Stratégies de réécriture}

\begin{frame}{}
\begin{itemize}
\item Pas de contrôles sur l'application des règles ;
\item Besoin d'une solution générique et expressive.
\pause
\item \vspace{0.4cm} Solution : langage de stratégie.
\end{itemize}
\end{frame}

\begin{frame}{Une stratégie}
Une stratégie est une fonction : $ terms * strategies \rightarrow terms $

\begin{itemize}
\item Altération des termes en fonction d'autres stratégies ;
\item Définition d'un nombre minimal de stratégies de bases.
\end{itemize}
\end{frame}

\begin{frame}[fragile]{Stratégies de bases}

{\fontsize{7}{7}

\begin{center}
    \begin{tabular}{ | l | p{6cm}|}
    \hline
    id() & identité \\ \hline
    fail() & toujours en erreur \\ \hline
    test(s) & identité si s reussi \\ \hline
    not(s) & identité si s ne reussi pas \\ \hline
    all(s) & applique s à tout les sous termes \\ \hline
    some(s) & applique s au maximum de sous termes \\ \hline
    one(s) & applique s à au moins un sous terme  \\ \hline
    s1 ; s2 & séquence de s1 puis s2 \\ \hline
    s1 +> s2 & application s1 ou s2 \\ \hline
    s1 + s2 & choix non-dét. : s1 et s2 en parallele \\ \hline 
    proj(n, s) & applique s au nième sous terme \\ \hline
    [r] & applique une règle r \\ \hline
    x & applique la stratégie dans x \\ \hline
    \end{tabular}
\end{center}
}
\end{frame}

\begin{frame}[fragile]{Stratégies par l'utilisateur}
\begin{verbatim}
strategy Topdown(s) :
  s ; all(Topdown(s))

strategy Bottomup(s) :
  all(Bottomup(s)) ; s
\end{verbatim}
\end{frame}

\begin{frame}[fragile]{Stratégies par l'utilistaur}
\begin{verbatim}
strategy Any(s) :
  test(s) +> one(Any(s))

# Any([some_rule])

strategy While(p, s):
  (test(p) ; s ; While(p, s)) +> id()

# si on a : rule [match] :  Foo(?x) -> Foo(?x)

# While(Any([match]), Do_thing)
\end{verbatim}

\end{frame}

\begin{frame}[fragile]{Stratégie de lambda-calcule}

\begin{verbatim}
# the main strategy : 
# - remove all pending substitutions
# - if it's an app ready for beta reduce, apply beta 
#  reduction then loop
# - if it's an app, reduce the right and left then loop
# - if it's a value do nothing
#
strategy ReduceAll :
    SubstAll ;
    (IsAppLambdaValue ; [beta]; ReduceAll) +>
    (IsApp ; Left(ReduceAll) ; 
      Right(ReduceAll) ; ReduceAll) +>
    (IsValue ; id())
\end{verbatim}

\end{frame}

\begin{frame}{}
\begin{itemize}
\item Système satisfaisant ;
\item Contrôle fin de l'application des règles ;
\item Bonne expressivité.
\end{itemize}
\end{frame}


%% utilisation du bouzin
%% Vincent
\section{Mode Interactif}
\begin{frame}{Directives Toplevel}



\end{frame}


%% démo
%% Pierre Roven
\section{Démonstration}

\begin{frame}
\begin{center}
\huge{Démonstration}
\end{center}
\end{frame}

%% Bilan

%% Pierre Inigo
\section{Bilan}
\begin{frame}
\frametitle{Système de réécriture}

\begin{block}{Étude des systèmes de réécriture}
\begin{itemize}
\item Langage de réécriture et typage ;
\item Algorithme de pattern matching non linéaire ;
\item Hash-consing ;
\item Multiple représentation d'une même donnée ;
\item Langage de stratégie.
\end{itemize}
\end{block}

\end{frame}

\begin{frame}
\frametitle{Travail d'équipe}

\begin{itemize}
\item Répartition des tâches via rallydev ;
\item Utilisation d'un dépôt git.
\end{itemize}

\begin{block}{Réunion développeurs}
\begin{itemize}
\item Environ 1 fois par semaine, bonne fréquence ;
\item Trop rigide par rapport au modèle Scrum au départ ;
\item Réunion et développement dans la même après-midi par la suite.
\end{itemize}
\end{block}
\end{frame}

\begin{frame}
\frametitle{Limites de Scrum}

\begin{block}{Environnement}
\begin{itemize}
\item Peu adapté à du développement sporadique ;
\item Diversité des périodes de pause (examens) ;
\item Diversité des emplois du temps ;
\item Malgré tout, proximité entre les membres de l'équipe.
\end{itemize}
\end{block}
\pause

\end{frame}

\begin{frame}
\frametitle{Limites de Scrum}

\begin{block}{Sujet}
\begin{itemize}
\item Partie recherche peu représentable via la méthode Scrum ;
\item Difficulté d'estimation ;
\item Néanmoins, fortement lié à l'implémentation dans notre cas.
\end{itemize}
\end{block}

\begin{block}{Réunion client}
\begin{itemize}
\item Peu de réunion réalisée mais productive ;
\item Interaction plus fréquente par mail (1 à 1).
\end{itemize}
\end{block}

\end{frame}


\begin{frame}
\frametitle{Conclusion}
\begin{itemize}
\item Système de réécriture fonctionnel ;
\item Sur des exemples complexes comme CCS ;
\item Documenté ;
\item Testé.
\end{itemize}
\end{frame}

\begin{frame}
\frametitle{Questions}
\begin{center}
\Large{Merci de votre attention !}
\end{center}
\begin{figure}[p]
  \centering
  \includegraphics[scale=0.6]{imports/question.jpeg}
\end{figure}
\end{frame}

\end{document}
