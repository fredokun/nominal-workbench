% Distributed under the MIT License.
% (See accompanying file LICENSE.txt)
% (C) Copyright NoWork team

\documentclass[12pt,a4paper]{article}

\usepackage[utf8]{inputenc}
\usepackage[american]{babel}
\usepackage[T1]{fontenc}
\usepackage{color}

\title{NoWork\\
User manual}
\author{NoWork development team\\[2em]}
\date\today

\begin{document}
\maketitle


\section{Introduction}
%% Pierre
What is NoWork and why is it interesting with regards to other rewriting system?

\subsection{Rewriting system}
%% Rémy
In computer science, we often define an operation of transition. From a
given state we go to an other state by this operation. In arithmetics,
we call this the \emph{evaluation} : we apply the calculus rules to
compute the result of a formula. In language semantics, we also call
this \emph{reduction}. A reduction step leads an expression in a given
state in a new state. Morally, in languages, the reduction operation
consists in replacing subterms of an expression by other terms.

A rewriting system is a method to describe a language by defining a
\emph{set of terms}, and the reduction operation on this language which
is a \emph{set of reduction rules}. 
To rewrite a term, you must first find the subterms which \emph{match} the left-hand side of one of the rules. Then you \emph{replace} it by the right-hand side of the choosen rules. It
could be indeterministic. In a rewriting system, it's possible to implement a lot of concepts like
$\lambda$-calculus $\beta$-Reducton, Peano's arithmetics, Milner's CCS,
De Morgan's Laws, etc. 

  For instance, if we considere Peano's arithmetics, 
  given the rule (\texttt{Sub(Add(X))} $\rightarrow$ \texttt{X}),
  the term \texttt{Sub(Add(0))} rewrites in \texttt{0}.

\subsection{Getting started}
%% Vincent
How do we install it (with opam or from source), which program do I need to launch and where is it?

\section{\texttt{nowork --help}}
%% Vincent
Presentation of the nowork executable and the options.

\section{System language}
%% Matthieu
Presentation of the language (everything but the :command). A tutorial, example-oriented.

\subsection{Semantics}
%% Rémy

\section{Strategy language}
%% Roven

\section{Interactive language}
%% Vincent
Presentation of the interactive language (:command).

\section{Example}
%% Roven
Present full program.

\subsection{lambda Calculus}

\subsection{CCS}


\end{document}
