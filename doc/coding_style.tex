\documentclass{article}

\include{verbatim}

\title{Nominal Workshop : \\
  Coding style conventions}

\begin{document}

\maketitle

\section{Lexical Conventions}

\subsection{Naming}

\medskip

\begin{itemize}
\item \textbf{Function and variable names}: Lower cases, with underscore to 
  separate the words (e.g. open\_file).
\item \textbf{Object name}: Use the camel case (e.g. FileOpener).
\item \textbf{File names}: The name of any files (e.g. .ml or .mli) must be in
  lower case.
\item \textbf{Module name}: same as functions and variables, with first letter
  in capital. 
\item \textbf{Sum types and exceptions}: should be written as module's names.
\end{itemize}

\subsection{General}

\medskip 

\begin{itemize}
\item \textbf{Indentation}: With two-spaces (don't use tabulation, set your
  editor to use spaces.)
\item \textbf{Line width}: in general, avoid more than 80 characters. It is
  easier to read.
\item \textbf{Tuples}: the comma should always be followed by a space, to ease
  reading. For example, \textsf{(a, b)}, not \textsf{(a,b)}.
\item \textbf{Lists}: the list constructor \textsf{::} is always preceded and
  followed by a space.
\item \textbf{Parenthesis}: avoid unnecessary parenthesis, even for tuples when
  it is not needed. To enclose a imperative-style sequence, prefer \textsf{begin
    .. end}.
\end{itemize}

\section{Indentation conventions}

To avoid indentation problems, simply use a good editor (Tuareg for emacs, for
example). However, here are some conventions to respect:

\subsection*{Pattern-matching}

In the \textsf{match .. with} case, cases should be aligned with the
\textsf{match}:

\begin{verbatim}
let eval ast =
  match ast with
  | Int i -> ...
...
\end{verbatim}

The vertical bar for the first case can be avoided, but the beginning of the
case must be indented once.

In case \textsf{function} is used over \textsf{match .. with}, the cases must be
indented once. For example:

\begin{verbatim}
let eval ast = function
  | Int i -> ...
...
\end{verbatim}

And not:


\begin{verbatim}
let eval ast = function
               | Int i -> ...
...
\end{verbatim}

\subsubsection*{Pattern-matching in a pattern-matching}

\medskip

Use \textsf{begin .. end} around the inner pattern-matching. The use of
parenthesis should be avoided.

\subsection*{The \textsf{let .. in} clause}

The \textsf{in} should be placed at the end of the expression, even in case of
long expressions. For instance:

\begin{verbatim}
...
let r = fun_call
  arg1
  arg2 in
...
\end{verbatim}

Consecutive \textsf{let .. in} share the same indentation:

\begin{verbatim}
...
let a = ... in
let b = ... in
...
\end{verbatim}

And not:

\begin{verbatim}
...
let a = ... in
  let b = ... in
    ...
\end{verbatim}

Actually, any expression after the \textsf{in} should be indented as its
\textsf{let}.


\end{document}
